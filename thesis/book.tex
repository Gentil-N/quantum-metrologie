\documentclass[10pt]{report}
\usepackage[utf8]{inputenc}
\usepackage{graphicx}
\usepackage[a4paper,width=160mm,top=17mm,bottom=17mm]{geometry}
\graphicspath{ {images/} }

\usepackage{amsmath}
\usepackage{amssymb}
\usepackage{esvect}
\usepackage{stmaryrd}



\begin{document}
\title{Title}
\author{Neven Gentil}
\date{May 2024}
\maketitle

\begin{abstract}
My very nice abstract
\end{abstract}

\chapter{First chapter}
\section{Optical Cavity}
\paragraph{}
We firstly introduce one of the most common object of quantum optic, namely, the \textit{optical cavity}. It is mainly constitued of two mirrors mounted perpendicularly to the optical axis at a certain distance from each other. The idea behind this simple setup is to select a discret range of wavelengths defined by the space between the mirrors. 

Indeed, one can modelize the cavity as an one-dimensional box, then apply the wave equation $ \square \vv{E} = 0$ where $\square$ is the d'Alembertian operator and $\vv{E}$ the electric field's amplitude, together with the Maxwell equation $\nabla \cdot \vv{E} = \partial _x E = 0$. As the wave must vanish at the boundaries of the box (meaning that the energy cannot propagate outside of the box's region) one would find that the solutions are linear combinations of sinuzoidale functions, each of them having a spatial frequency multiple of $\frac{\pi}{L}$ where $L$ denotes the length of the box. Thanks to the relation $\omega = ck$ we see that the angular frequency is also quantized.\footnote{R.Loudon derives rigorous equations in sections 1.1 and 1.2 of \textit{Quantum Theory of Light}.} It is important to note that the argument is valid for higher dimensions.

With a more pragmatic point of view, one can think of this wavelength selection as an interferometer. In fact, let's imagine for a while that in front of the sun, we emplace two parallel mirrors, fast enough to capture a beam of sunlight then we stop the clock before further propagation. Mathematically, the waves which are differents from the selective frequency vansih instantaneously, because of the presence of the cavity. However, physically, the full spectrum of the sun still resides between the two mirrors. Then if we run the clock again, the light is reflected from the back mirror. As we captured a continous stream, the dephased reflected light\footnote{Fresnel equations dictate the light's behavior between different media.} encouters the incoming light: interference happens and after a couple of round-trips, only the in-phase waves (i.e with a frequency $\frac{n\pi}{L}, n\in \llbracket 0, +\infty \llbracket $) are interfering constructively whereas the out-of-phase ones vanish. That is why the device created by C.Fabry and A.Pérot in 1899 is called an \textit{interferometer} as, in their case, the light hit the mirrors with an angle, creating the famous pattern of fringes. Equivalently, this category of instrument is often named \textit{resonator} as the amplitude of the electric field is maintained to its initial value in the case of resonance with the cavity or decreases in other cases: it is by construction an \textit{almost-}perfect\footnote{As we will see later, each resonant frequency owns a bandwidth in contrary to a harmonic oscillator where the ladders are well defined.} harmonic oscillator and obeys to similar equation from a classical or quantum point of view.

In real experiments, the device is static and one prefers to let the light enter partially on one side to feed the cavity, while partially exiting through the other side to collect the selected waves. In that way, we use \textit{partially reflective} mirrors parametrized by a reflection coefficient $R$ and a transmission coefficient $T$ obeying to the relation $R + T = 1$ (meaning that all the absorbed energy must be re-emitted by the mirror).



\end{document}